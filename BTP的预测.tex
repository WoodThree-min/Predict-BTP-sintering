\documentclass[12pt,a4paper]{article}
\usepackage{fontspec}
\usepackage{setspace}
\usepackage{xeCJK} 
\defaultfontfeatures{Mapping=tex-text}
\usepackage{xunicode}
\usepackage{xltxtra}
\usepackage{amsmath}
\usepackage{amsfonts}
\usepackage{amssymb}
\usepackage{graphicx}
\usepackage{subfigure}
\usepackage[left=2cm,right=2cm,top=3cm,bottom=3cm]{geometry}
\usepackage{float}

\renewcommand{\baselinestretch}{1.25}

\begin{document}

\begin{center}
\begin{LARGE}
\textbf{BTP的预测}
\end{LARGE}
\end{center}
\textbf{预测目标:预测未来10分钟、15分钟的BTP状态}

由前分析,我们可得到一段时间(约两小时)内模拟的BTP轨迹,即 
$\widehat{z_{B}}$. 每一个数据点代表每五秒获得的一个新的估计值。
为了达到预测目标,首先,以12个数据点为一组,计算每分钟BTP的估计值,即每组数据的 $\widehat{z_{B}}$的平均值,作为新的样本 $ Y $进行分析和预测, $ Y $在未来第10阶和未来第15阶的预测即为未来10分钟、15分钟的BTP预测。

\section{$Y $的分析}

结论:Y具有较强的趋势性,不具有随机数据的特征,ARIMA模型等不适用于Y的拟合和预测。

分析方法:小波分析方法和傅立叶方法。

\subsection{自相关函数和偏自相关函数}


\subsection{小波分析方法}
 选用小波基coif2对$ Y $做小波分析。
 
 首先在Level 3的水平上对$Y$做小波分解,$ Y $被分解为A3, D3, D2, D1.
 D为分解得到的随机项,从图1的振幅来看,随机项的影响相对较小。
 并且由于原始数据$ Y $的计算依赖于优化模型,此处随机项具有为随机性,不做为分析的重点。
 主要分析A3的拟合估计。
\begin{figure}[H]
 \centering
 \includegraphics[height=3in,width=6.5in]{coif(2,3)_1.png}\\
 \caption{小波分解(coif2,Level 3 )\quad}
\end{figure}

由图2-1可知,A3可以较为准确的拟合$Y$的趋势,过滤噪音,但仍出现明显的拟合错误(比如在第110分钟左右)。图2-2的原始系数显示,Level 3 的窗口大小约为10。
\begin{figure}[H]
 \centering
 \includegraphics[height=3in,width=6.5in]{coif(2,3)_2.png}\\
 \caption{趋势拟合(coif2,Level 3 )\quad}
\end{figure}

而图3的傅立叶变换谱显示能量在频率0.1前集中,即窗口大小约为15时,振幅特征可被有效提取。
\begin{figure}[H]
 \centering
 \includegraphics[height=3in,width=6.5in]{coif(2,3)_3.png}\\
 \caption{残差分析(coif2,Level 3 )\quad}
\end{figure}

由以上分析,可选择小波分析的窗口为15分钟。在此,我们用小波基coif2在Level 4的水平上对$Y$做小波分解,由图4和图5-1可知,A4可以准确的拟合$Y$的趋势(也许稍显粗糙)。图5-2也标明Level 4的窗口大小约为15。
\begin{figure}[H]
 \centering
 \includegraphics[height=3in,width=6.5in]{coif(2,4)_1.png}\\
 \caption{小波分解(coif2,Level 4 )\quad}
\end{figure}

\begin{figure}[H]
 \centering
 \includegraphics[height=3in,width=6.5in]{coif(2,4)_2.png}\\
 \caption{趋势拟合(coif2,Level 4 )\quad}
\end{figure}

图6的傅立叶变换谱也证明了窗口大小设定为15的充分性。
\begin{figure}[H]
 \centering
 \includegraphics[height=3in,width=6.5in]{coif(2,4)_3.png}\\
 \caption{残差分析(coif2,Level 4 )\quad}
\end{figure}

\subsection{傅立叶方法}

对整个时间序列信号$ Y $做傅立叶变换,如图7,明显可以得出,能量集中在频率0.1之前,同上述图3、图6中的残差傅立叶谱图一致,表明:窗口大小约为15时,振幅特征可被有效提取。

\begin{figure}[H]
 \centering
 \includegraphics[height=3in,width=6.5in]{fft_1.png}\\
 \caption{傅立叶谱\quad}
\end{figure}

设定窗口大小为15,对上述谱进行傅立叶重构(傅立叶逆变换),由图8可以明显看出,重构的时间序列准确拟合了原数据$ Y $的趋势,消除了毛刺。
\begin{figure}[H]
 \centering
 \includegraphics[height=3in,width=6.5in]{fft_2.png}\\
 \caption{窗口为15的傅立叶重构\quad}
\end{figure}

\section{$ Y $ 的预测}


基于以上分析,我们将用局部常系数模型(Locally Constant Mean Model)来进行该时间序列分析,用15分钟内的$ Y $的平均或加权平均值作为未来的预测值,同时计算出其以2倍标准差为误差范围的预测区间。
\begin{figure}[H]
 \centering
 \includegraphics[height=3in,width=6.5in]{predict_1.png}\\
 \caption{未来10分钟的BTP状态\quad}
\end{figure}

\begin{figure}[H]
 \centering
 \includegraphics[height=3in,width=6.5in]{predict_2.png}\\
 \caption{未来15分钟的BTP状态\quad}
\end{figure}





\end{document}
